\chapter{El Estado moderno no es otra cosa que la religión de Estado}
\label{cha:el-estado-moderno}

El Estado moderno nace para poner fin a la guerra de religiones. Tiene un cuerpo que captura y que produce. Se vale de una neutralidad central para poner fin a la guerra de religiones y es condición de garantía para la pluralidad ética, para la coexistencia de las religiones si se supeditan a la concepción capitalista del trabajo. En ese sentido, el Estado supera todas las formas anteriores de religión. \emph{La religión de Estado} captura el deseo a través de la economía. De acuerdo con Tiqqun, el fin de la guerra de religiones está en la conversión y no en la asunción de las formas de vida. La hostilidad propia de la guerra de religiones se transfiere a una hostilidad general hacia el otro en su forma de ciudadano.

En ese sentido, el súbdito y el soberano son dos caras del mismo proceso de estamentación. La apropiación del deseo tiene como objeto producir el interés de reproducir un rol en la jerarquía y suprimir o dar cauce a cualquier pulsión. Ambos viven para contener a cada \emph{forma de vida} en su Yo incorporando representaciones del propio cuerpo y de la psique. Estos sujetos son los agentes del aparato de captura y se encargan de darle sentido al trabajo y a la producción para que su finalidad sea siempre el plusvalor, la utilidad. La curialización de los guerreros en las cortes es buen ejemplo \autocite{tiqqunIntroduccionGuerraCivil2008}. Eso pasa con la militancia en los tiempos de la política folk, cargada de manuales de etiqueta y simbolismos sin demasiado alcance tecnomaterial (votos, recursos, influencia o base social).

Retomando la introducción, el neorreaccionario aboga por el despotismo ilustrado. Esta señal es muestra de la configuración imperial del Estado moderno contemporáneo, que es a su vez, su crisis. A continuación doy una perspectiva a partir de \emph{Tiqqun} sobre cómo ha ocurrido este proceso y a qué responde la existencia histórica y la crisis del Estado moderno cuyo fin ha sido poner fin a la guerra de religiones. Para el Comité Invisible, la Historia del Estado moderno comienza como una posición frente a la guerra de religiones y la época contemporánea se caracteriza por una crisis profunda de la religión de Estado. A lo largo de este capítulo haré un bosquejo del concepto del Estado sin más a partir de su funcionamiento a través de máquinas y aparatos. Lo anterior me servirá para explicar por qué el Estado moderno no es otra cosa sino la religión de Estado y cuáles son las repercusiones de este proceso a nivel molecular. Es decir, cómo el Estado afecta a las formas de vida y cómo el Estado se erige como el principal productor de subjetividad llevando la guerra a la vida interior de cada \emph{forma de vida} para que en el dominio público prevalezca la neutralidad.

En este capítulo~\ref{cha:el-estado-moderno} desarrollaremos la cuestión de los aparatos y agentes del Estado que fungen como su aparato de captura del deseo.

\section{El Estado se vale de una máquina de guerra capitalista y su propio aparato de captura}
\label{sec:el-estado-se-vale-de}

Comenzaré con explorar el concepto de Estado sin más. De acuerdo con Deleuze y Guattari, en un sentido noológico, el Estado es \emph{Urstaat}, es el \enquote{Logos}, \enquote{La Palabra}, un espacio estriado en cuya configuración se crea la correspondencia cosa-palabra. Es decir, el mundo es un todo fragmentado por el Estado, por la palabra, por la correspondencia unívoca que hay entre cosas, enunciaciones y acciones \autocite{deleuzeMilMesetasCapitalismo2002}. Lo opuesto al Estado sería un espacio liso, donde el lenguaje se resbala de las cosas y el nexo cosa-palabra es más casual. En una metáfora, el Estado produce círculos ideales en oposición a un redondel, que es una forma circular imperfecta, concreta. Asume como real aquello configurado a través del lenguaje y no a través de la materialidad, de lo corporal, de su conformación técnica. En un sentido ontológico, las esencias difusas, aquellas que no produce el Estado, son continuas, extraen de las cosas una determinación que más que la coseidad, es la de \enquote{corporeidad}, implica un espíritu de cuerpo. Para desarrollar este argumento, los autores explican que si los orígenes del número se remontan a la cifra, a la necesidad de contar, el Estado plantea un número numerante, que da al número la correspondencia cosa-palabra. Busca regir ahí para dar al pensamiento una forma de interioridad, una realidad intrínseca, y darle a esta interioridad una forma de universalidad: \enquote{la finalidad de la organización mundial es la satisfacción de los individuos razonables dentro de los Estados particulares libres}. De ese modo, el fin del Estado es convertir al pensamiento en máquina de guerra, ponerlo en relación inmediata con el afuera. Es \emph{imperium} y república al mismo tiempo \autocite[pp.~380, 381, 397]{deleuzeMilMesetasCapitalismo2002}.

Para entender lo anterior me resulta muy útil el concepto de agenciamiento, que es la correspondencia cosa-palabra, donde la significación procede de agentes colectivos y de un conjunto de enunciados que disponen a los cuerpos. El agenciamiento o la instancia concreta del Estado es siempre histórica, compuesta por agentes que significan los hechos históricos y las relaciones entre objetos para ciertos fines. El Estado moderno como manifestación del \emph{Urstaat}, cuya etimología \emph{st} significa lo estático, se vale de dos medios principales para sus operaciones sobre la subjetividad y sobre las instituciones: la policía y la publicidad. El fin de estos medios es transformar toda \emph{forma de vida} en ciudadanía, en deuda al E\emph{st}ado. Antes de desarrollar los medios del Estado moderno, señalaré algunas cuestiones sobre el funcionamiento del Estado sin más.

El Estado opera a través de una ciencia de las velocidades o dromología, una forma de saber que analiza el poder desde sus operaciones técnicas y desde sus velocidades. Más allá de las posiciones contractualistas, el Estado sin más, como categoría ontológica, dispone a la interioridad subjetiva al afuera, a la vida psíquica de las formas de vida al ritmo de una máquina de guerra social que permite al Estado expandirse como una categoría del pensamiento. Este sentimiento de totalidad es descrito por el Comité Invisible en la Introducción a la guerra civil como una neutralidad central, como la idea de lo uno, como una configuración estriada del espacio. De esta manera, el Estado configura una forma total de lo \emph{real}, con tiempos, velocidades y disposiciones de objetos a fines explícitos y sobre todo, diseñados para la guerra. La forma en que el Estado construye sus herramientas es siempre como armas destinadas a la captura. Es la ruta ingenieril, una posición técnica \autocite{comiteinvisibleNuestrosAmigos2015}. El Estado domina a través del arte militar, de la guerra. Entre sus tácticas, estría, es decir, fragmenta, el espacio sobre el que reina o usa espacios lisos para la comunicación de espacio estriado. Es decir, se apropia de los flujos y continuidades para reproducir una lógica estriada del mundo \autocite{deleuzeMilMesetasCapitalismo2002}. La dromología de Paul Virilo señala que en ese sentido, el poder político del Estado es \enquote{polis}, policía, red de comunicación. Ve al Estado como \emph{échangeur routier} (ingeniero) y a la dromología como el sistema proyector y proyectil. La velocidad es la cuestión central al analizar las cualidades técnicas de armas y herramientas. De ahí que el nacimiento de los Estados-nación requirió de la captura y de la puesta en marcha de una economía de guerra donde las agrupaciones de mercenarios nómadas se convirtieron en ejércitos al servicio de un soberano. Esto nos muestra la capacidad del Estado de producir redes de comunicación que son en sí la arquitectura del control, de la apropiación de los caminos para el cobro de una renta que sostiene los flujos económicos del Capital.

En un sentido sociohistórico, El Estado se explica más por la progresión de fuerzas económicas y políticas que como resultado de una guerra con frentes y motivos definidos, como algo determinado por la contingencia que produce una narrativa de correspondencia palabra-cosa-Historia. Parece que esta ficción nos muestra que hay una relación de interior y exterior como alcances de la ley del Estado, pues interioridad y exterioridad, máquinas y aparatos, están en coexistencia y competencia, interactúan. Cuando Marx habla del ejército de reserva, no solo se refiere a aquellos trabajadores que ya forman parte del proletariado sino a todos aquellos con una potencia capturable. En ese sentido, el Estado, al tener como fin capturar y explotar las potencias de las formas de vida y orientarlas a un arreglo productivo, transforma todo trabajo de subsistencia --un entendimiento instrumental del trabajo como herramienta-- a trabajo bélico, deuda, trabajo hacia un afuera. Dispone las técnicas del trabajo como arma, como apropiación de la potencia vital para una velocidad de aceleración que tiene como fin la posibilidad de la guerra. Lo que caracteriza al Estado moderno capitalista es que la posibilidad de la guerra está todo el tiempo destinada al aniquilamiento, a la hostilidad absoluta \autocite{tiqqunIntroduccionGuerraCivil2008}.

El Estado pone fin a la guerra de religiones a través de un uso estratégico de armas y herramientas orientadas al capitalismo, a la extracción de valor. El Estado moderno es una máquina de guerra capitalista. Retomando a Marx, el modo de producción asiático es una máquina de guerra despótica. El proyecto de la modernidad consiste justamente en la \emph{transfiguración de la máquina de guerra nomádica en la guerra como la finalidad de toda forma maquínica}. De ahí el papel que tiene la ingeniería y el giro de la economía mercantilista a la financiera. El pragmatismo fue lo necesario para que el Estado protestante se expandiera al incorporar el parásito capitalista y se trata de cierta forma de hacer las cosas que es capaz de hacerse más operativa a sí misma.

En lo que respecta a la máquina social que produce subjetividades, la operación de apropiación se constituye en buena medida gracias a la neutralización de la guerra y a la curialización del guerrero. El nacimiento y profesionalización de los ejércitos nacionales rompe con la lógica de negociación de hordas de bárbaros que ven a la guerra como una cuestión suplementaria. La disciplinarización de la guerra significa una alienación de las formas de violencia, rompe con la posibilidad del conflicto entre pares y de la preparación ritual para la disputa corporal. El desarrollo militar se manifiesta en la célebre frase de Keynes cuando dice que su teoría solo puede probarse en tiempos de guerra \autocite{renshawWasThereKeynesian2016}. A partir del papel tan significativo que tuvo la guerra durante el siglo pasado, durante mi investigación encontré muchas teorías que explicaban la formación del Estado desde posiciones historicistas, colonialistas e incluso descolonialistas. Sin embargo, lo que yo buscaba era una forma de entender la cuestión que me permitiera analizar al Estado desde los procesos que lo configuran, con la intención de analizar sus movimientos como los miraría un ingeniero. Desde ese punto de vista orientado a la acción crítica es que me cuestioné cuáles son los circuitos económicos del Estado moderno.

A continuación presento algunas definiciones de utilidad para analizar las operaciones del Estado moderno como forma histórica total. Para hacerlo, me he valido de dificultosas lecturas de \emph{Capitalismo y esquizofrenia}, particularmente de los apartados \enquote{1227 - Tratado de nomadología: la máquina de guerra} y \enquote{7000 a.J.C. - Aparato de captura}, del artículo \emph{La máquina social} de Carlos Rojas Osorio \autocite{GILLESDELEUZEMAQUINA} así como de una posición aceleracionista de izquierda desarrollada en la revista \emph{tripleampersand} del \emph{The New Center for Research \& Practice} en la ciudad de Nueva York, conocida como \#altwoke \autocite{AltWokeCompanion2017}. En el siguiente apartado desarrollo el modelo de máquina de guerra capitalista y del aparato de captura del Estado. Parto de la siguiente premisa:

\subsection{Bajo el capitalismo, la máquina de guerra capitalista produce mientras el aparato de captura del Estado sustrae}
\label{sub:bajo-el-capitalismo-la-máquina-de-guerra-capitalista}

En este apartado expondré algunas cuestiones relativas a las máquinas. Deleuze y Guattari cambian radicalmente el concepto de deseo que había sido mantenido casi siempre ---con excepción de Spinoza y Nietzsche--- como simple carencia de algo. Por el contrario, el deseo es producción, voluntad de poder; afecto activo diría Spinoza. El deseo como carencia es un concepto idealista, en realidad de raigambre platónica. Deleuze, en cambio, logró ver que el deseo produce realidad.

En la máquina deseante ven Deleuze y Guattari ante todo flujos. Toman la idea de Lawrence: la sexualidad es flujo. Todo deseo es flujo y corte. Flujo de esperma, de orines, de leche, etc. Freud descubrió este flujo de deseo. Ricardo y Marx descubrieron el flujo de producción, el flujo de dinero, el flujo de mercancías; todo ello como esencia de la economía capitalista. Lo que caracteriza al sistema es la apropiación del producto por parte del capital. También Lutero descubrió la religión como fenómeno estrictamente privado, muy acorde con la nueva economía del capital. La producción de deseos es inconsciente, como bien vio Freud. Pero en lugar de la producción de deseos Freud instauró un teatro burgués, porque instauró en el inconsciente la mera representación. En cambio, el deseo tiene poder para engendrar su objeto. Las necesidades derivan del deseo, y no al revés. Desear es producir, y producir realidad. El deseo como potencia productiva de la vida. Por eso Deleuze y Guattari ubican el interés de clase en el inconsciente. Y entre ambos entablan relaciones diferenciales. Por ejemplo, un revolucionario puede serlo al nivel de clase, del interés de clase y, sin embargo, estar dentro de una estructura autoritaria desde el punto de vista libidinal. Viceversa, un revolucionario de la máquina deseante podría ser ajeno a la revolución de la máquina social. Para Deleuze lo decisivo es que el eslabón más débil, el momento de verdadera ruptura llega por el lado de la máquina deseante. Desde luego, no son las condiciones sociales efectivas. Es como la libertad y el determinismo de Kant. Hay un determinismo social, las condiciones objetivas de la máquina social; pero hay también el momento discontinuo de la producción y el deseo. Y es éste, apoyado en las condiciones objetivas, lo que posibilita el acto de libertad supremo que es la producción del deseo. El campo social se carga de una producción represiva o bien de un deseo revolucionario. Este último lo denominan nuestros autores el individuo esquizo. Y el tipo de análisis psicológico lo llaman esquizoanálisis. Sea en la producción represiva que en el esquizo la máquina social es la misma. El esquizo es el productor universal. El sujeto es también producción. Los autores califican a su psicología de materialista: introducir el deseo en el mecanismo social, pero también introducir la producción en el deseo. El esquizo no cree en el yo. La teoría de Freud depende demasiado del yo.

Frente a esta posibilidad de libertad, la economía capitalista organiza la necesidad, la escasez, la carencia. El objeto depende de un sistema de producción que es exterior al deseo. El campo social está atravesado por el deseo. La máquina social es también producción deseante. \enquote{Sólo hay deseo y lo social, nada más}. Paranoia y esquizofrenia son los dos polos de la máquina social. El paranoico tiende a Edipo, a la ley, al orden, al código, al significante. Se proyecta imponiendo el orden, arraigando la autoridad, tiranizando. En cambio, el esquizo constituye la línea de fuga de la máquina social. Busca la producción de la máquina deseante. Nada hay más revolucionario para la máquina social que la máquina deseante. El deseo es primero y fundamental; tiende también a decodificar las estructuras sociales y no coincide con la decodificación que lleva a cabo el capital. Freud se fijó en la represión, pero no logró relacionarla con la represión general que se lleva a cabo siempre en la máquina social. Fue Reich quien asoció correctamente la represión general con cada una de las máquinas deseantes. Por medio de la familia la estructura autoritaria de la sociedad se prolonga hasta sus más íntimos engranajes. El problema de la política lo planteó Spinoza: ¿por qué combaten los seres humanos por mantenerse en la servidumbre como si fuera su salvación? Lo que sorprende es que los explotados no se rebelen o que los hambrientos no roben\autocite{deleuzeMilMesetasCapitalismo2002}.

Deleuze y Guattari hablan de tres tipos de máquina social: la máquina salvaje, la máquina bárbara o despótica y la máquina capitalista. La máquina salvaje está fundada sobre la tierra, sobre el cuerpo de la tierra. Es territorial. Sobre el cuerpo de la tierra inscribe sus insignias, que son las de la alianza y la filiación. Lo decisivo son las relaciones de parentesco, lo que no quiere decir que lo económico sea marginal. El parentesco domina las relaciones primitivas pero por razones económicas. La máquina bárbara coincide con lo que Marx denominó el modo de producción asiático. Aparece el Estado, ya completo y en su forma general que fundamentalmente no cambiará ni siquiera hasta el socialismo oriental (ruso-chino); vieja herencia que se prolonga por milenios. El estado es la máquina despótica y recubre los viejos territorios fundados sobre el cuerpo de la tierra. El estado organiza un sistema de producción que unifica el anterior sistema territorial. Decodifica sus antiguos códigos y los recodifica en el lenguaje del despotismo estatal. Para Deleuze el gran corte de la historia está en la aparición de la máquina estatal. La sociedad no se funda en el don, como creía Marcel Mauss; se funda en la deuda. Lo propio de la máquina capitalista es hacer la deuda infinita. El capitalismo no puede proporcionar un único código que abarque todo el campo social; al contrario, es decodificador. Pero en lugar de un código instaura una axiomática abstracta de cantidades monetarias. La axiomática se caracteriza por la fecundidad de sus axiomas de base. La axiomática capitalista se distingue porque puede agregar siempre nuevos axiomas.

La máquina deseante es un sistema de producir deseos; la máquina social es un sistema económico-político de producción. Las máquinas técnicas no son independientes ni exteriores a la máquina social. Cada técnica forma parte esencial de la máquina social. La tecnología capitalista es esencial al sistema de explotación capitalista. Son grandes máquinas las que son usadas para la explotación de grandes masas de trabajadores.

De acuerdo con Deleuze y Guattari, la máquina de guerra es aquella formación cuyo fin \emph{suplementario} es la guerra. La máquina de guerra capitalista es aquella para la cual la guerra es su sentido total, la realidad plena de su existencia. La máquina de guerra libera un vector Velocidad \autocite[p. 399]{deleuzeMilMesetasCapitalismo2002}. Los motores ideales son:

\begin{description}
  \item[Trabajo:] choca con resistencias, actúa sobre el exterior, se consume, gasta y debe ser renovada su energía. Velocidad relativa.
  \item[Acción libre:] vence resistencias, actúa sobre el cuerpo móvil en sí mismo, no se consume en su efecto y es continuo. Es \emph{perpetuum mobile}.
\end{description}

La diferencia entre un aparato de captura y una máquina de guerra está en las armas y herramientas, que se distinguen según su uso: destruir hombres o producir bienes. En un sentido físico, el arma es centrífuga y la herramienta es centrípeta \autocite[p. 398]{deleuzeMilMesetasCapitalismo2002}. Una herramienta encuentra resistencias a vencer, mientras que un arma se encuentra con respuestas a evitar e inventar. Tiqqun considera al Estado moderno, es decir, a su instancia histórica, en su carácter de narrativa que configura la memoria, como una ``unidad mecánica, como máquina, como artificialidad consciente''. En términos maquínicos, el Estado transforma el deseo en interés a través de la mercancía. Agamben señala que la mercancía es al mismo tiempo inutilizable y sagrada \autocite{agambenHomoSacer1998}. El Estado es enemigo de todo lo que quiera desbordarlo [p. 394]\autocite{deleuzeMilMesetasCapitalismo2002}.

Por otro lado, el \emph{aparato de captura} se refiere a un acuerdo psíquico implícito y silencioso entre agentes y agenciamientos, para hacerse de los medios de producción: la tierra, el capital y el trabajo. Estas operaciones económicas del Estado capitalista con agentes humanos son también producciones de subjetividad de estamentos para el mantenimiento de la máquina de guerra. En el caso de los agentes humanos estas producciones de subjetividad pueden comprenderse también como un tributo que consiste en trabajo reproductivo.

El agenciamiento que produce el Estado capitalista sobre las formas de vida humanas muestra un rasgo propio de los dispositivos del Estado capitalista: No hay una necesidad intrínseca de cierta tecnología. Más bien la tecnología evoluciona con la máquina social de la que forma parte. La máquina deseante no se da sin la máquina social, y viceversa. La naturaleza también es máquina deseante. Por ello Deleuze y Guattari hablan de la continuidad Naturaleza-hombre.

\subsection{El capitalismo es un virus ultra edípico}
\label{sub:el-capitalismo-es-un-virus-ultra-edípico}

Otra de las características económicas del Estado moderno tiene que ver con el nacimiento y el desarrollo del mercantilismo como teoría económica y de su posterior transformación en la compleja sociedad financiera global. En contraste con las investigaciones de Marcell Mauss sobre las sociedades del don, que sin lugar a dudas ha abierto un campo de investigación muy prolífico para otras posibilidades más allá de la economía capitalista, la sociedad moderna se funda en la deuda. Continuando con el argumento de la subsección~\ref{sub:bajo-el-capitalismo-la-máquina-de-guerra-capitalista}, en este apartado sostengo que el capitalismo es un virus ultra edípico en la medida en que reafirma viejas estructuras sociales heredadas de fases precapitalistas para poder codificar el deseo.

Como punto de partida y siguiendo a Deleuze, considero que el capitalismo es una axiomática de flujos descodificados. Es decir, el capitalismo es un intérprete de flujos de producción deseante libre y se vale del Estado para constituir agenciamientos cosa-palabra cuyo fin es la extracción de valor. El capitalismo decodifica los viejos códigos fundados sobre la máquina despótica pero los territorializa a su favor, dentro de su poderosa axiomática. El neurótico se queda en los códigos establecidos, queda instalado en las viejos territorios, en los residuos que han quedado al salto de la máquina bárbara a la máquina del capital. El perverso explota la palabra y crea territorios artificiales. El esquizo emprende la línea de fuga de todo territorio codificado, lo desterritorializa todo. Marx había observado muy bien que el capitalismo arrolla con todo lo que antes se consideraba sagrado, lo decodifica todo. El esquizo se mantiene en el límite. Mezcla los códigos. La esquizofrenia es la producción deseante como límite de la producción social.

Más allá de la concepción freudiana, Edipo es una entidad metafísica. Para entender la dominación que ejerce es preciso, como Kant, hacer una crítica de la metafísica. Para el colectivo ANON se trata de una revolución trascendental pero materialista: denunciar el uso ilegítimo de Edipo \autocite{AltWokeCompanion2017}. El revolucionario desconoce a Edipo, no reconoce padre, ni dios. El inconsciente es huérfano, no necesita inconsciente como productor de sentido. El esquizoanálisis es político y revolucionario. El inconsciente es roussoniano. El hombre es naturaleza. El deseo es revolucionario, cuestiona el orden establecido. El deseo es activo, agresivo, artista (creación libre) , productivo, conquistador. La literatura es también esquizofrenia, proceso de producción sin fin. La única literatura es la que hace estallar el superego.

Por otro lado, el capitalismo lo privatiza todo. La esencia del capitalismo se halla en dos fenómenos complementarios: desterritorialización y descodificación. El capital se apropia cada vez más de territorios; se apropia del campo, del artesanado, del comercio y finalmente de la industria. El capital lo desterritorializa todo. Pero al mismo tiempo lo decodifica todo: la religión, la moral, las creencias; todo sucumbe al impulso del capital. Este impulso anulador de códigos y apropiador de territorios es universal en el capitalismo. El capitalismo es, por ello, \emph{lo universal de toda sociedad}. Pero se decodifica para someter nuevamente a la axiomática potente del capital. La televisión, por ejemplo, nos da todo, la sociedad y el capital a la vez. No es necesario salir afuera. Todo el sistema del capital está ahí en la pantalla televisiva. En el capitalismo se unifica la memoria y la reproducción modificando la explotación del ser humano. En la sociedad lo esencial es marcar y ser marcado. Se trata de memoria codificando sobre cuerpos; escritura corporal, memoria social.

Los rangos pertenecen a la máquina territorial, las castas a la máquina despótica imperial, las clases a la máquina social. Sin embargo, los tres tipos de sociedad no transcurren de forma lineal. Hay más bien, una secuencia de estratos. La máquina despótica estatal se monta sobre la máquina de la tierra, recodificándola. A su vez la máquina del capital se monta sobre la vieja máquina estatal, descodificándola y apropiándose de los viejos territorios. El Estado en la sociedad capitalista reúne códigos, aglutina a los desperdigados por el poder desterritorializador y decodificar del capital. Como vio Marx, las clases son el negativo de las castas y los rangos. Dada esta taxonomía, los sistemas del socialismo real no son otra cosa que capitalismo de Estado. Edipo es el déspota. Todos los sistemas autoritarios han prolongado el autoritarismo familiar, edipiano. A su vez el Estado y la máquina social prolongan también el autoritarismo edipiano. Así, la tendencia autoritaria de la revolución está desde sus inicios.

Marx pensaba que llegaría el momento en que la clase obrera no tendría nada que perder, habría tocado fondo, habría perdido todo territorio y todo código, estaría en la nada, entonces daría el todo por el todo y la revolución sería posible. Por otro lado, Nietzsche pensaba que el desierto crece y que hay que asumir el nihilismo. Es verdad que tras la posmodernidad no hay validez alguna, todo se ha vuelto inválido. Hay que apropiarse de esta pérdida de todo criterio, de este nihilismo y llevarlo hasta las últimas consecuencias. Hay que asumir valientemente la pérdida del supuesto \enquote{mundo verdadero} (el mundo inteligible del platonismo). Y sólo así, en el desierto absoluto, quizá algo nuevo pueda llegar a valer \autocite{GILLESDELEUZEMAQUINA}. Así pues, no hay posibilidad de movimiento emancipatorio alguno si no se completa con una amplia consideración de la estructura familiar. Todo ello realizado con un instrumental conceptual nuevo y creativo. El valor de Capitalismo y esquizofrenia es que recoge ideas marxianas y critica ideas freudianas para enderezarlo hacia un esquizoanálisis. Deleuze y Guattari dirán que tienen más importancia las líneas de fuga del capitalismo, como ciertas formas de arte, ciertas tendencias dentro de la ciencia. Hay ciencia paranoica y hay ciencia esquizo. Los conflictos raciales de muchos países, los conflictos de nacionalidad, el feminismo, son líneas de fuga.

\section{El Estado moderno instaura la religión de Estado para poner fin a la guerra de religiones}
\label{sec:el-estado-moderno-instaura}

\epigraph{La historia de la formación del Estado en Europa es la historia de la neutralización de los contrastes confesionales, sociales y de otro tipo en el seno del Estado.}{\emph{Carl Schmitt} en
\autocite{tiqqunIntroduccionGuerraCivil2008}.}

Me situaré por un momento en la teoría de Tomás Hobbes, comúnmente considerado como fundador del Estado moderno. A la parafernalia jurídica de John Stuart Mill o de John Locke precede un pensador de avanzada que es capaz de decir que el poder de la Iglesia debe someterse al del Estado. Hay una razón para ello: el miedo, particularmente el miedo a la muerte es la condición que inaugura la religión de Estado al configurar una experiencia religiosa de obediencia y de temor a la Ley. Esta doctrina de vida como autopreservación, como cuidado de la longevidad, suena a una transformación del temor de Dios judeocristiano.

\begin{quote}
  \enquote{\emph{El temor de Jehová es el principio de la sabiduría,\\ 
  Y el conocimiento del Santísimo es la inteligencia.}}\\
  Proverbios 9:10.
\end{quote}

La pluralidad ética es la promesa del Estado moderno, la neutralidad central que niega el conflicto. Tiqqun define la guerra civil como el libre juego de las formas de vida, mientras que la empresa de neutralización, la ficción metafísica de que existe algo como La Ciencia, La Razón, La Justicia, cuyo papel es legítimo a priori para hacer juicios de carácter universal, son formas de enmascarar ese deseo de Estado, el deseo de totalidad, del Uno, de lo real como plenamente estriado, plenamente sometido al Estado, como el número numerante \autocite{tiqqunIntroduccionGuerraCivil2008}. En este apartado desarrollaré en qué sentido el Estado moderno es la religión de Estado.

\subsection{La soberanía proviene el cuerpo de Cristo}
\label{sub:la-soberanía-proviene-el-cuerpo-de-cristo}

El soberano hereda su poder de Dios. La operación de Hobbes al mover el poder divino de la iglesia al Estado da al soberano un papel canónico. El soberano moderno construye esa vida religiosa desde la publicidad. Pareciera que en realidad la doctrina del Estado moderno es la secularización del dogma representado por el símbolo de Nicea. Hobbes le consagra un capítulo en el Leviatán, puesto que la teoría de la soberanía personal descansa sobre la doctrina que hace del Padre, del Hijo y del Espíritu Santo, tres personas de Dios en el sentido de quien juega su propio papel o el de otros. Esta doctrina también permite definir al soberano como actor de aquellos que han decidido designar a un hombre o una asamblea para asumir su personalidad, de manera que cada cual se confiesa o reconoce como el autor de todo lo que habrá hecho o hecho hacer, lo que concierne a la paz y seguridad común, aquel que ha asumido así su personalidad. \autocite[pp. 59-97]{tiqqunIntroduccionGuerraCivil2008}. El Estado moderno se concibe como la parte de la sociedad que no forma parte de la sociedad y que, por eso mismo, se halla en condiciones de representar. Como una continuación de la teología iconófila de Nicea, donde Cristo como ícono no manifiesta la presencia de Dios sino su ausencia esencial, su retiro sensible, su irrepresentabilidad. El soberano personal es aquel de quien la sociedad civil, ficticiamente, se ha retirado \autocite[p. 63]{tiqqunIntroduccionGuerraCivil2008}.

\begin{quote}
  \enquote{\emph{La misión del soberano (sea un monarca o una asamblea) consiste
en el fin para el cual fue investido con el soberano poder, que no es
otro sino el de procurar la seguridad del pueblo.}} \autocite{hobbesLeviatan2007}.
\end{quote}

Por seguridad no se entiende aquí una simple conservación de la vida, sino también de todas las excelencias que el hombre puede adquirir para sí mismo por medio de una actividad legal, sin peligro ni daño para el Estado. Esto significa que mientras no se dañe el orden, las excelencias que al hombre se le ocurran les son permitidas. En ese sentido, el Estado es legítimo porque:
\begin{enumerate}
  \item le da sentido a mi vida sometiendo mi voluntad y
  \item me permite establecer contratos por la paz con otros para mi vida familiar y para satisfacer mis necesidades.
\end{enumerate}

En este sentido, el \emph{telos} del Estado es preservar la paz y crear un marco que garantice las libertades necesarias para que la gente satisfaga sus necesidades.\footnote{Hay una doble cuestión porque Hobbes parte del conservadurismo y del orden para llegar a la libertad \autocite[pp. 31-58]{tiqqunIntroduccionGuerraCivil2008}. Hayek parte de lo opuesto, pero en cierto sentido considera el orden como una cuestión igualmente relevante.} Esta historicidad propia de la modernidad plantea una ruptura con la visión de la edad antigua que consiste en un cambio en la concepción de la guerra y la paz. Para la Antigüedad, el estado normal es el estado de guerra, al que viene a poner fin una paz, mientras que para las personas modernas, la paz es el estado normal, que es perturbado por la guerra. Esta cuestión sobre la guerra civil me hizo pensar en un paralelismo entre el Concilio de Nicea por el que se declaró la paz entre diferentes sectas cristianas y el arreglo territorial y de linaje que surgió después de la Paz de Westfalia, que consistió en el arreglo estatal-nacional de vastos territorios en Europa.\footnote{Y que es semejante, a su vez, al reparto territorial e inicio de movimientos independistas o antimperialistas que siguieron después de la Segunda Guerra Mundial y continuaron hasta la caída del muro de Berlín. Para efectos prácticos, asumiré que la transición entre la posmodernidad y la época contemporánea comienza con la disolución de la URSS y termina con los atentados a las Torres Gemelas en la ciudad de Nueva York.} Respecto a la guerra, Hobbes señala:

\begin{quote}
  \enquote{\emph{Cada hombre debe esforzarse por la paz, mientras tiene la esperanza de lograrla; y cuando no puede obtenerla, debe buscar y utilizar todas las ayudas y ventajas de la guerra.}} \autocite[capítulo 14]{hobbesLeviatan2007}.
\end{quote}

Una ley sólo tiene valor para Hobbes en la medida en que sujeta a una mayoría de sujetos de manera equitativa y resume esta regla en dos máximas: buscar la paz y seguirla, y defendernos a nosotros mismos por todos los medios posibles. Según nuestra interpretación, la rebeldía se justificaría por sí misma cuando el Estado no garantiza la paz y es legítima si por ella misma puede instaurar un orden que permite preservar la paz.

\subsection{La fe y la Crítica son formas de neutralidad}
\label{sub:la-fe-y-la-crítica-son-formas-de-neutralidad}

El hecho de que Kant mencionara una premisa que también pronunciaba Federico II y que postulaba \enquote{Razonad tanto como queráis y sobre todo lo que queráis, ¡pero obedeced!}, muestra la escisión entre público y privado a partir de una barbaridad: la crítica. Y es que la crítica instaura al espacio político moralmente neutro de la Razón de Estado \autocite{deleuzeMilMesetasCapitalismo2002}, el espacio moral, políticamente neutro, del libre uso de la razón. Así, esta visión instaura el dominio de la publicidad. Y la publicidad es la neutralización de la crítica, pues el pensamiento solo deviene máquina de guerra cuando la crítica acontece como exterioridad, como fe en la acción, como proposición, como una ética. El encapsulamiento de la crítica es la apropiación de las prácticas burguesas de clase sobre la palabra-crítica, su disolución en mera discursividad. Aquello fue lo que hizo decir a Alexis de Tocqueville en 1856, que en Francia toda pasión pública se disfraza de filosofía (en El Antiguo Régimen y la Revolución) y es violentamente ahogada en la literatura. La ficción central de los Tiempos Modernos es una fantasmagoría conocida como: \emph{neutralidad central.} Esta se puede comprender como conceptos metafísicos tales como la Razón, el Dinero, la Justicia, la Ciencia, el Hombre, la Civilización o la Cultura. Esta neutralidad es un movimiento fantasmagórico que plantea tal centro como éticamente neutro. La operación estatal de neutralización instituye dos monopolios dependientes: el monopolio de lo político y el monopolio de la crítica. Esta mancuerna da origen a lo que llamamos el monopolio de la moral, de la ética de una \emph{societas}. Sin embargo, la ficción del Estado como ficción del Uno, cuenta con medios reales de los que ha dispuesto para su propia supervivencia: la policía y la publicidad. Así, rompe con la posibilidad de un afuera, de una multiplicidad de órdenes, para exclamar: \enquote{fuera de mí, solo desorden}. Las ficciones de la modernidad consisten en un proceso de movilización sin fin. Es decir, no se trata de un estadio donde uno está instalado sino un imperativo de modernización, de \enquote{flujo tenso, crisis a crisis, al final vencido únicamente por nuestra laxitud y nuestro escepticismo} \autocite[p.~33]{tiqqunIntroduccionGuerraCivil2008}. La vida eterna es la Historia universal, la memoria del Estado universal como proyecto teológico. Parece que heredara de la Iglesia el espíritu totalitario, por pretender abarcar la vida desde el nacimiento hasta la muerte.

\subsection{El Estado universal es, a su vez, una condición necesaria para la pluralidad ética}
\label{sub:el-estado-universal-es-una-condición-necesaria-para-la-pluralidad-ética}

La célebre precisión de Kant de que la suya no era una época ilustrada sino de ilustración resume concretamente la temporalidad del Estado. Así, la continuidad del Estado moderno depende de una guerra imparable, implacable e interminable, contra la guerra civil. El Estado precisa reafirmar su existencia a cada instante pues solo aquello que es ficticio tiene la necesidad de constituirse como real.\footnote{En todo caso, lo real está relacionado con los agenciamientos del pensamiento-palabra sobre un espacio liso, que fungen como la variabilidad de las variables, como cuerpo sin órganos. Es decir, el pensamiento-palabra está vivo, deviene. (Tratado de Nomadología en \autocite{deleuzeMilMesetasCapitalismo2002}).} En Ideas para una Historia Universal en clave cosmopolita, Kant señala que el 2º principio dice que la razón se desarrolla en la especie, no en el individuo. Así, se implanta la ficción de la Humanidad como Naturaleza. La Naturaleza opera a través del antagonismo, la insociable sociabilidad de los hombres, la instauración de una hostilidad que pone en riesgo permanente a la sociedad. Incluso agradece a la Naturaleza por la envidiosa vanidad que nos hace conformar un cuerpo social a partir del interés. En el quinto principio \autocite[p.10]{kantQueEsIlustracion2009}, señala que el mayor problema para la especie humana es la instauración de una sociedad civil que administre universalmente el Derecho. Entre más libertad, mayor el antagonismo de sus miembros. Necesitamos una Constitución civil perfectamente justa \footnote{Quizá de aquí viene la inspiración de Hardt y Negri sobre el poder constituyente \autocite{hardtImperio2005}.}. Séptimo: se necesitan reglamentaciones sobre relaciones interestatales. Kant reserva al Estado la posibilidad de la guerra. El \emph{telos} de los antagonismos de Estado es un Estado cosmopolita de la seguridad estatal pública. El hombre ilustrado \autocite[p.21]{kantQueEsIlustracion2009}, ha de ascender poco a poco hasta los tronos y tener influencia sobre sus principios de gobierno. La Naturaleza prescribe un Estado cosmopolita, universal. Kant propone una historia universal, que no sería solo como una novela, sino algo que encauzaría la ambición de los jefes de Estado hacia un recuerdo glorioso.

\subsection{La guerra de religiones termina con la conciencia del miedo, por un mero instinto de preservación}
\label{sub:la-guerra-de-religiones-termina}

\epigraph{\textsc{Cuius regio, eius religio}.\\ (\textsf{A tal rey, tal religión}.)}{\emph{Frase latina}, de moda durante la Paz de Augsburgo.}

Para Hobbes, el primer gran contractualista, las pasiones que inclinan a los hombres a la paz son el temor a la muerte, el deseo de las cosas que son necesarias para una vida confortable y la esperanza de obtenerlas por medio del trabajo. El Estado hace la paz subyugando las voluntades de los hombres, creando sentidos. Podríamos explicar esto como que el Estado subyuga las voluntades para evitar que haya voluntades que choquen cuando las personas quieren la misma cosa, dando un marco para que las voluntades de la gente no choquen. El Estado es, en doble sentido, un contrato más y aquel contrato que posibilita cualquier otro contrato \autocite{hobbesLeviatan2007}. Es decir que el Estado subyuga las voluntades y es la condición que posibilita cualquier otro contrato. Hobbes llega a comentar que no pueden existir contratos entre aquellos que tienen el mismo derecho sobre todas las cosas \autocite{hobbesLeviatan2007} y que estos contratos solo pueden adquirir validez cuando otro individuo restringe y subyuga las voluntades naturalmente irrestrictas de ambos contratantes.

\begin{quote}
  \enquote{\emph{[\ldots] durante el tiempo en que los hombres viven sin un poder común que los atemorice a todos, se hallan en la condición o estado que se denomina guerra; una guerra tal que es la de todos contra todos. Porque la guerra no consiste solamente en batallar, en el acto de luchar, sino que se da durante el lapso de tiempo en que la voluntad de luchar se manifiesta de modo suficiente. [\ldots] existe continuo temor y peligro de muerte violenta; y la vida del hombre es solitaria, pobre, tosca, embrutecida y breve}.}~\autocite[Capítulo 13]{hobbesLeviatan2007}.
\end{quote}

A partir de lo anterior sostengo que el Estado, en cuanto ente, tiene un \emph{telos}, una finalidad. La particularidad del Estado moderno es que su \emph{telos} constituye la instauración definitiva de la necesidad de una neutralidad central. Es decir, el Estado se asume como finalidad en sí mismo. En la monarquía Hobbes plantea que la voluntad del pueblo descansa en el \emph{cuerpo} del soberano. El Leviatán en cuanto Estado es un monstruo pacificador que opera a través de la alienación de todo componente social en la vida de las personas, dando nacimiento a la tecnología como arte efectiva del control. Y su táctica ha sido unir grupos de mercenarios nómadas para construir un ejército cuyo combustible deja de ser pulsionar y se transforma en heroico, en la realización del Edipo.

Hobbes ha creado al Estado. Y la cualidad que comparten ambos en cuanto voluntades es siempre la de devorar el mundo, de consumirlo, de hacer de él un cuerpo homogéneo de soldados. Hobbes es, en este sentido, un enemigo de la guerra. La legitimidad significa la capacidad de hacer reconocer y aceptar una acción. Dado que el Estado es otro plenipotenciario según Hobbes, para analizar la legitimidad del Estado se necesita responder ¿por qué voy a someter mi voluntad a otro? La respuesta según nuestro autor es por miedo a morir. De modo que someto mi voluntad a un soberano que nunca me hará daño y, por si fuera poco, dará sentido y cauce a mi vida; y me permitirá hacer contratos con otros en calidad de iguales. Bajo este esquema es que logrará darme un cauce. Así, no someto mi voluntad, sino que satisfago mi interés y mis necesidades.

Su comprensión de la moralidad no tiene un \emph{telos} o un \emph{summum bonum} como lo señalan Platón o Aristóteles. El fin de la moralidad es únicamente la preservación natural del individuo y de la especie. La tercera ley de la naturaleza argumenta que el pacto aumenta nuestra probabilidad de sobrevivir. En la tercera y cuarta parte del Leviatán argumenta en contra de la visión cristiana de obedecer a la conciencia de uno y se decanta por la obediencia absoluta hasta el límite en que nuestra vida peligre. Así, Hobbes pone fin a la guerra de religiones para instaurar la religión de Estado. Mientras nuestra vida no corra riesgo, hay que actuar de acuerdo a los derechos del soberano. En las conclusiones del Leviatán, Hobbes señala que sus planteamientos consideran tanto la naturaleza humana como las leyes naturales. La naturaleza humana legitima el contrato porque el ser humano es un ser irritable y espinoso, regido por la competencia, la timidez o inseguridad, y la gloria.

Paradójicamente, señala también que la guerra no empieza por grandes desacuerdos sino por pequeñas discordias, a lo que Freud llamaría \emph{narcicismo de las pequeñas diferencias} \autocite{freudTresEnsayosTeoria2017}. La moralidad, comenta Hobbes, no es algo a lo que los hombres aspiren como un fin para sí mismo. Es externa y es necesaria para vivir en sociedad y gozar de privilegios como el arte, la cultura y el lujo. De modo que cada individuo es inmoral por naturaleza y se adapta para extraer los valores extrínsecos a la moralidad a través de los pactos. En ese sentido, el contrato social es un imperativo moral cautelar. Podemos decir que, en este sentido, el Estado es legítimo por su capacidad de crear un orden y un régimen a través de la violencia y el miedo.

La soberanía puede ser rastreada en la delegación voluntaria de la autoridad por los ciudadanos. El Leviatán es, además, una extrapolación de la autoridad paternal del monarca, presente en todo núcleo familiar. El contrato está hecho para que no nos matemos los unos a los otros.

\begin{quote}
  \enquote{\emph{En esta guerra de todos contra todos, se da una consecuencia: que nada puede ser injusto. Las nociones de derecho e igualdad, justicia e injusticia están fuera de lugar. Donde no hay poder común, la ley no existe [\ldots]}} en \autocite[Capítulo 13]{hobbesLeviatan2007}.
\end{quote}

En este Estado Hobbes concluye que por lo mismo no puede existir ni lo justo, ni lo injusto ni la propiedad ni la industria \autocite{hobbesLeviatan2007}, pero tampoco pueden existir relaciones o contratos particulares, leyes, convenciones, pues no habría ningún incentivo o castigo para ninguno de los involucrados a permanecer en dicho contrato de manera organizada.

Con esto he tratado de señalar cómo el Estado moderno ha surgido para poner fin a la guerra de religiones. Pareciera que el Estado moderno no es otra cosa sino la instauración de la religión de Estado. Rescatamos que algunos de los puntos claves serán el análisis de la guerra y la soberanía como condiciones para el desarrollo de un Estado universal o universalista que garantice la pluralidad ética, que reproduzca la teleología milenarista de la Iglesia católica. El Estado moderno es religioso porque su fe se basa en una teleología de la Historia como motor del progreso y no como el campo de batalla de distintas fuerzas batiéndose, del conflicto. \emph{La Historia universal es la nueva épica del Estado moderno.}

El Estado moderno experimenta tres facetas con equivalentes en acontecimientos históricos. El Estado absolutista en la guerra de religiones, como Estado liberal en la lucha clases y como Estado providencia frente al Partido imaginario. El fracaso moderno, la historia misma del Estado moderno, radica en el proceso mismo. \footnote{Es importante resaltar que la transición del Estado absolutista al Estado liberal ocurre por la instauración de lo que Foucault llamaría la república fenoménica de los intereses \autocite{tiqqunIntroduccionGuerraCivil2008}.}

La Reforma rompe con la organicidad de las mediaciones consuetudinarias, para introducir una doctrina que profesa una separación tajante entre la fe y las obras, del reino del Dios y del mundo, del hombre interior y del hombre exterior. Así, por efecto de las querellas entre sectas, las religiones introducen sin quererlo la idea de pluralidad ética. Todavía en este momento, la guerra civil es concebida por quienes la incitan como algo que habría de encontrar su término en la conversión. Es decir, aquellos señores, operadores de la guerra de religiones, no conciben nunca la asunción de las \emph{formas de vida}. En ese sentido, el Estado moderno es la religión del Estado, la disolución de toda religión por la pluralidad ética. El Estado moderno se engendra a sí mismo a partir de la nada, al alienar la técnica política del tejido ético y reconociéndola como el fundamento de su soberanía. Sin embargo, para sobrevivir, tiene que constituir lo real como el Uno. La paradoja consiste en que necesita enunciarse para ser real, concebir todo espacio como espacio estriado y a toda \emph{forma de vida} como súbdito, sin dejar espacio a nada fuera del Leviatán. El Estado moderno es la religión del Estado, la disolución de toda religión por la pluralidad ética. De aquí el gesto melancólico de autopreservación del que Hobbes escribe con tanto ahínco. Entre más nos alejamos de ese gesto, más pierde su sentido. Esa calma desesperanza se sintetiza en la expresión \autocite[pp. 31-58]{tiqqunIntroduccionGuerraCivil2008}.

\begin{quote}
    Así, el Estado universal nace como garante de la pluralidad ética.
\end{quote}

En síntesis, la religión de Estado es el fin de la guerra de religiones. El Estado es el ordenamiento y la alienación de la técnica política, de las técnicas de producción. Produce una posición religiosa sobre el trabajo a través de la representación en la ecología psíquica. El Estado moderno tiene máquinas y aparatos en coexistencia para reorientar el deseo. Lo que el Estado moderno hizo tras la paz de Westfalia fue establecer al sistema de propietarios como el sistema de gobierno combinado diferentes arquitecturas de producción conectadas en red para la preservación del estado de las cosas. Si existe un enemigo a señalar es el sistema de propietarios y los diferentes aparatos de captura del Estado que producen representaciones de sentido, aspiraciones y otros agenciamientos del Estado en la producción social de deseo.

\section{La religión de Estado es la guerra de una persona contra sí misma}
\label{sec:la-religión-de-estado-es}

\epigraph{\textsc{Homo homini lupus}\\ (\textsf{El hombre es el lobo del hombre}).}{\emph{Locución latina}.}

El Estado adquiere su forma social a través del interés económico. El interés es la manifestación mercantil de un proceso de producción de deseos que a cada momento configura la voluntad del deseo al interés y rompe el vínculo social. Para comprender la evolución del Estado profundizaré a continuación en la relación entre este y la economía. El capitalismo como modo de producción no es sino una de las operaciones históricas del Estado y consiste en su fase productiva más veloz. El Comité Invisible señala cómo el Estado instaura la hostilidad a través del miedo como una forma de plantearse como garante necesario de cualquier relación entre los individuos. Esta operación es realizada a través de la economía como la instauración del reino de la necesidad. En resumen, el Estado, a través del interés, en su modalidad histórica de Estado moderno, transforma todo deseo en una operación de abstracción donde la mercancía se vuelve el objeto último de deseo y de subjetivación. El Estado, en cuanto ente, tiene un \emph{telos}, una finalidad. La particularidad del Estado es que su \emph{telos} constituye la instauración definitiva de la necesidad, de una ficción de unidad, de la reducción total del mundo a lo Uno, a La Ley. Es decir, el Estado se asume como finalidad en sí mismo \autocite{tiqqunIntroduccionGuerraCivil2008}.

\begin{quote}
  \enquote{\emph{La era burguesa clásica ha colocado así los principios cuya aplicación ha hecho del hombre eso que conocemos: la agregación de una nada doble, la del \enquote{consumidor}, ese intocable, y la del \enquote{ciudadano} (¿qué puede ser más ridículo, en efecto, que esa abstracción estadística de la impotencia que se insiste en seguir llamando ciudadano?).}} \autocite{comiteinvisibleTeoriaBloom}.
\end{quote}

De este modo, el modelo de ciudadano deviene ciudadano-consumidor. El Estado prepara con efectividad a sus siervos en el terreno del deseo para reproducir la operación primigenia de la economía: la apropiación de toda nuda vida a través del trabajo asalariado, condición inaugural del capitalismo. Es necesario aclarar que el capitalismo es el modo de producción consecuente a la instauración primigenia de la necesidad como realidad económica y que la legitimidad del Estado proviene siempre del despliegue de su aparato y dispositivos, que operan produciendo deseos, sentidos de realidad, y siendo el mediador entre los contratos particulares, para preservar la paz y para satisfacer necesidades. La economía es la condición de posibilidad y al mismo tiempo, la realidad plena del proyecto moderno. Se trata de un embrujo, de una magia que instaura la necesidad para reducir todo deseo al mero interés. La economía tiende a homogeneizar los afectos a través del dinero, pero este no es lo mismo que el mercado, que bien podríamos definir como la red de transacciones, a una velocidad determinada, de bienes y de imágenes. Toda imagen, todo deseo, está mediado por una abstracción (la moneda) que da vida a las cosas y las transforma en mercancías, esa fantasmagoría que ya señala Marx en su teoría del valor. El Comité Invisible señala cómo el Estado instaura la hostilidad a través del miedo como una forma de plantearse como garante necesario de cualquier relación entre los individuos. Esta operación es realizada a través de la economía como la instauración del reino de la necesidad. En resumen, el Estado, a través del interés, transforma todo deseo en una operación de abstracción donde la mercancía se vuelve el objeto último de deseo y de subjetivación.

En realidad, la sociedad es la ruptura de todo vínculo social, pues la hostilidad configurada a través de la mercancía busca a toda costa evitar el vínculo social pues este significaría la ruptura con dispositivos como la familia, el género o la identidad, necesarios para continuar con el modo de producción de la economía para el Estado. La sociedad moderna es la mercantilización de todo vínculo social. Opera a través de la moneda y depende de un gran Contrato para existir, aunque éste realmente sea el agenciamiento de, más allá de sus producciones, un cierto modo de aprehender lo real. El Estado moderno instaura el reino de la necesidad a través del interés económico, que reorienta la voluntad del deseo al interés y rompe el vínculo social. El ciudadano es consumidor, es un súbdito del Estado que cede sus deseos y pulsiones a cambio de un sentido total basado en la alienación mercantil.

\subsection{Los sujetos del Estado moderno son el súbdito y el soberano}
\label{sub:los-sujetos-del-estado-moderno-son-el-súdito-y-el-soberano}

\begin{quote}
  \enquote{\emph{¿Qué eres tú para mí? ¿Por qué habría de actuar de acuerdo contigo en vez de conforme a mi propia voluntad, pues si yo no te obstaculizo, tú puedes actuar según la tuya y no la mía?}}.\autocite[p.~63]{hobbesCiveElementosFilosoficos2016}. 
\end{quote}

Podríamos señalar dos caras del proceso de subjetivación, una que produce al soberano y otra que da como resultado al súbdito. La subjetividad correspondiente al Estado moderno es la del consumidor, pues el ciudadano es un súbdito del Estado que cede sus deseos y pulsiones a cambio de un sentido total basado en la alienación mercantil. Sin embargo, este proceso nunca es completo. Durante la formación del Estado moderno, el soberano encarnado era el símbolo de la represión de las formas de vida por estar sometidas a su placer. \emph{El Estado moderno es por tanto antes que nada la constitución de cada cuerpo en Estado molecular}. En consecuencia, cuanto más se constituyen las sociedades en Estados, más se incorporan sus sujetos a la economía. Se vigilan a sí mismos y entre ellos, controlan sus emociones, sus movimientos, sus inclinaciones, y creen poder exigir de los otros la misma contención. Se cuidan de no abandonarse nunca ahí donde podría serles fatal, y se montan un pequeño rincón de opacidad donde tendrán todo el placer de soltarse. Se encadenan a sí mismos los unos a los otros, contra cualquier desbordamiento.

Cada cuerpo, para llegar a ser sujeto político en el seno del Estado moderno debe pasar por el proceso de fabricación que lo convertirá en tal: debe comenzar por dejar de lado sus pasiones, impresentables, sus gustos, irrisorios, sus inclinaciones, contingentes, y debe dotarse, en lugar de esto, de intereses, que son con certeza más presentables y hasta representables. En consecuencia, cada cuerpo, para llegar a ser sujeto político, debe empezar por proceder a su autocastración en sujeto económico. Idealmente, el sujeto político se habrá reducido a un puro voto. La función esencial de la representación que una sociedad da sobre sí misma es la de influir sobre la manera en que cada cuerpo se representa a sí mismo y, por tanto, sobre la estructura psíquica. Es decir que el sujeto económico da sentido al sujeto político.

El soberano es el depositario de todos los poderes y voluntades particulares que en tanto que existe, debe de ofrecer la seguridad de todos sus súbditos. Un soberano sólo es legítimo en la medida en que posea de facto la capacidad de proteger a sus súbditos. La legitimidad no descansa en la razón ni en las leyes, sino en la fuerza que puede ostentar un soberano, y el temor que justifica lo suficiente a sus súbditos de alienar su voluntad \autocite[p.~39]{tiqqunIntroduccionGuerraCivil2008}. La curialización de los guerreros ofrece el ejemplo arquetípico de esta incorporación de la economía cuyos jalones van del código de comportamiento cortés del siglo \textsc{xii} hasta la etiqueta de la corte de Versalles, primera realización de envergadura de una sociedad perfectamente espectacular donde todas las relaciones son mediadas por imágenes, pasando por los manuales de civismo, de prudencia y de \emph{saber vivir}. La violencia, y muy pronto todas las formas de abandono que fundaban la existencia del caballero medieval se encuentran lentamente domesticadas, esto es, aisladas como tales, desritualizadas, excluidas de toda representación y finalmente reducidas por el cotilleo, el ridículo, la vergüenza de tener miedo y el miedo de tener vergüenza. Es por la difusión de este auto-constreñimiento, de este terror al abandono, que el Estado ha llegado a crear al sujeto económico, a contener a cada uno en su Yo, es decir, en su cuerpo, a extraer de cada \emph{forma de vida}, nuda vida. La guerra civil se ha refugiado en todos, el Estado moderno ha puesto a cada cual en guerra contra sí mismo.

La guerra de todos contra todos es más bien la indigente ética de la guerra civil y la guerra civil es el libre juego de las formas de vida, el principio de su coexistencia. La economía es la neutralización de la guerra y la instauración del reino de la hostilidad. La contención es la represión de la propia violencia. Dice el Comité Invisible que si las sociedades tradicionales conocían el robo, la blasfemia, el parricidio, el rapto, el sacrificio, la afrenta y la venganza, en la modernidad, la violencia es aquello de lo que hemos sido desposeídos, y aquello de lo que hace falta ahora reapropiarse \autocite{tiqqunIntroduccionGuerraCivil2008}.

Lo enunciado es la muerte de aquello que se supone ausente, pues solo si se asume como ausente se puede conjurar algo cuyo propósito es instaurar lo ausente. Así, toda institución es la muerte de su razón de ser y la sociedad es la ruptura de todo vínculo social.

El súbdito es antecedente de la ciudadanía y el soberano quien hereda el poder de Dios. El proceso por el que el súbdito accede a la explotación es para castrase, para renunciar a su propio deseo y convertirse en un sujeto económico.

He tratado de mostrar como el trabajo y los agentes de recaudación del Estado para el capital se encargan de reproducir la cadena de producción y el régimen de deseo jerarquizado para la reproducción social del ejército de reserva. En el siguiente capítulo me concentrare en analizar la transformación y pulverización de estos mecanismos, que bien podríamos sintetizar en la transición de la publicidad al Espectáculo y de la policía al Biopoder. Para aproximarme a la cuestión de la complejidad, introduciré el concepto de cibernética para hacer una reflexión profunda de la transición a la que Deleuze se referiría como sociedades disciplinarias a sociedades de control. Sin embargo, es necesario que entendamos que las formas disciplinarias, de control y de cansancio, por señalar algunas, tiene que ver con una adaptabilidad psíquica del Estado capitalista para la extracción de valor. La nuda vida es reconocimiento para la productividad de cada área de la vida.