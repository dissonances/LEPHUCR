¿Cuáles son las posibilidades de actuar en en el momento más incierto y más absurdo de la Historia? Mientras cientos de activistas marchan religiosamente exigiendo al \emph{gobierno} (como si hubiera, de hecho, algo ahí que responda al llamado) un mundo más libre, más justo, más ecológico, los males propios de nuestro tiempo están presentes en todo momento en nuestra vida. Se trata de una configuración tecnológica de la realidad, una disposición de las cosas para que la gente actúe y responda frente a ellas de cierto modo. En la era de información, el rostro del totalitarismo se confunde en un virus capitalista que ciertamente ha dominado la nuda vida: en la sociedad no hay vida así sin más. Corporaciones, máquinas y aparatos ejercen su poder sin ninguna resistencia organizada capaz de proponer una alternativa universal al \emph{estado} actual de las cosas.

Vivimos una imperceptible guerra civil en la que el enemigo a vencer no está ni siquiera en el montón de idiotas que rigen las corporaciones y los gobiernos nacionales sino en una religiosidad que profesa simpatía por la dominación mercantil. La Iglesia de esta fe se llama Estado moderno y en su génesis se encuentra la posiblidad misma de su destitución. En este texto contribuyo a analizar la relación entre Estado, capitalismo y mercados, así como el funcionamiento de la forma Estado como si se tratase de una ingeniería inversa, de un hackeo, pues el Leviatán funciona a partir de máquinas y de aparatos de captura, tiene agentes pagados que se encargan de cumplir diversas funciones alrededor del proceso de extracción de valor. Además de agentes, posee ejércitos, fuerza de trabajo que alimenta cada día la interacción entre mercados, que a su vez se vuelve el único medio por el cual es posible adquirir los medios de subsistencia. Sin embargo, el virus capitalista hace de cualquier necesidad un producto y del deseo, interés.

No vivimos por sino a pesar del capitalismo, toda relación social está mediada por las mercanías.

Con la caída del modelo de producción fordista y la pulverización de la fábrica nos encontramos no solo frente a un momento de crisis en el modelo de trabajo del hombre-masa sino también frente a la acelerada pauperización de las resistencias y disidencias políticas. El Imperio se caracteriza por un reconocimiento cínico de la imposibilidad del proyecto universalista del Estado. Entonces, la policía y la publicidad, mecanismos del Estado moderno, se transforman en Espectáculo y Biopoder. Se borra todo grado de interioridad en la vida psíquica de las personas pues la Ley ha pasado a ser la norma, a estar completamente internalizada. Nuestras relaciones son espectáculares porque están mediadas a través de la representación cibernética y nuestro deseo bajo el Biopoder porque la industria farmacopornográfica se encarga de generar dinero de la enfermedad y de la frustración sexual. La cibernética es el arte del control, la disciplina propia de tiempos donde la información se vuelve la mercancía que más valor produce.

Frente al sinsentido contemporáneo, el \emph{sujeto} experimenta también una crisis profunda. El Comité Invisible llama Bloom al hombre-masa que se encuentra en el límite de la deserción, frustrado y castrado, enfermo de nihilismo y dispuesto al afuera. Sin embargo, la redención de esta forma masculina ha producido un Partido de la nada, la neorreacción, con una agenda tecnoutópica cuyo fin es hacer total el poder de la cibernética imperial. El Bloom es un hombre cínico que ha perdido incluso el significado que le daba el trabajo como recompensa frente a la alienación de verse reducido a un mero instrumento en una cadena de producción viva. Su fuerza vital es capturada por los agentes del Estaado encargados de producir el \emph{deseo de Estado} para alimentar una posición totalitaria y exterminista detrás de la que en realidad se oculta un profundo sentimiento infantil de abandono y represión sexual.

En este contexto, la acción crítica se inscribe en lo que yo llamo \emph{tecnocrítica}, análisis y develamiento de las condiciones tecnomateriales de posibilidad de las formas hablantes, de formas de opresión como el género, la raza o la clase. Solo así es posible entender los problemas económicos que hacen necesario al Estado y a sus agentes como soporte de información del capitalismo. Con la intención de encontrar una alternativa universalista, analizo diversos manifiestos y posiciones políticas contemporáneas alrededor del Partido y la hegemonía. Concluyo desarrollando algunas notas rumbo a la construcción del manifiesto para un Partido destituyente.
