% Options for packages loaded elsewhere
\PassOptionsToPackage{unicode}{hyperref}
\PassOptionsToPackage{hyphens}{url}
%
\documentclass[
]{article}
\usepackage{amsmath,amssymb}
\usepackage{lmodern}
\usepackage{iftex}
\ifPDFTeX
  \usepackage[T1]{fontenc}
  \usepackage[utf8]{inputenc}
  \usepackage{textcomp} % provide euro and other symbols
\else % if luatex or xetex
  \usepackage{unicode-math}
  \defaultfontfeatures{Scale=MatchLowercase}
  \defaultfontfeatures[\rmfamily]{Ligatures=TeX,Scale=1}
\fi
% Use upquote if available, for straight quotes in verbatim environments
\IfFileExists{upquote.sty}{\usepackage{upquote}}{}
\IfFileExists{microtype.sty}{% use microtype if available
  \usepackage[]{microtype}
  \UseMicrotypeSet[protrusion]{basicmath} % disable protrusion for tt fonts
}{}
\makeatletter
\@ifundefined{KOMAClassName}{% if non-KOMA class
  \IfFileExists{parskip.sty}{%
    \usepackage{parskip}
  }{% else
    \setlength{\parindent}{0pt}
    \setlength{\parskip}{6pt plus 2pt minus 1pt}}
}{% if KOMA class
  \KOMAoptions{parskip=half}}
\makeatother
\usepackage{xcolor}
\IfFileExists{xurl.sty}{\usepackage{xurl}}{} % add URL line breaks if available
\IfFileExists{bookmark.sty}{\usepackage{bookmark}}{\usepackage{hyperref}}
\hypersetup{
  hidelinks,
  pdfcreator={LaTeX via pandoc}}
\urlstyle{same} % disable monospaced font for URLs
\setlength{\emergencystretch}{3em} % prevent overfull lines
\providecommand{\tightlist}{%
  \setlength{\itemsep}{0pt}\setlength{\parskip}{0pt}}
\setcounter{secnumdepth}{-\maxdimen} % remove section numbering
\ifLuaTeX
  \usepackage{selnolig}  % disable illegal ligatures
\fi

\author{}
\date{}

\begin{document}

\hypertarget{convenciones-de-estilo}{%
\subsubsection{Convenciones de estilo:}\label{convenciones-de-estilo}}

\begin{itemize}
\tightlist
\item
  Para un uso preciso del lenguaje y frente a la convención de hablar
  del ``hombre'' para referirse al \emph{sujeto} universal, usamos
  siempre ``personas'' y ``gente'' para evitar el plural masculino como
  plural para cualquier multitud. En todos los casos donde se lee plural
  en femenino es para referirse a las personas como multitud a menos que
  se especifique o indique lo contrario.
\item
  Se dice que los autores de Tiqqun son un grupo de intelectuales que se
  conoce como ``los nueve de Tarnac'' y es muy probable que hayan
  escrito la colección de libros firmada por el Comité invisible,
  comenzando con ``La insurrección que viene''. Sin embargo, en realidad
  este hecho es incierto y quizá fueron más bien alumnas del primer
  círculo quienes continuaron la producción escritural. He decidido
  señalar al Comité invisible como autor de Tiqqun porque en diversas
  publicaciones se dice que el Comité invisible es el órgano
  revolucionario del Partido imaginario, que es la figura política que
  desarrollan a lo largo de toda su obra. Además, la dificultad de
  nombrar a Tiqqun como autor de la revista es que llevan el mismo
  nombre, además de que Tiqqun es también un concepto del judaísmo que
  puede traducirse como redención.
\end{itemize}

\end{document}
